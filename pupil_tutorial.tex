\documentclass{article}
\usepackage[utf8]{inputenc}
\usepackage{amsmath}
\usepackage{amssymb}
\usepackage{amsthm}
\usepackage{amssymb}
\usepackage{mathdots}
\usepackage{fancyhdr}
\usepackage{cancel}
\usepackage{MnSymbol,wasysym}
\usepackage{mathrsfs}
\usepackage{mathtools}
\usepackage[bookmarks]{hyperref}
\usepackage{hyperref}
\usepackage{listings}

\usepackage[margin=1in]{geometry}
\usepackage{fancyhdr}
\pagestyle{fancy}
\fancyhf{}

\usepackage{titling}
\setlength{\droptitle}{-8em}
% \predate{} \postdate{} \date{}
\usepackage{parskip}
\linespread{1.3}
\usepackage[default]{lato}
\usepackage{xcolor}
\definecolor{sectioncolor}{HTML}{0b2451}
\definecolor{subsectioncolor}{HTML}{000000}
\usepackage{sectsty}
\DeclarePairedDelimiter{\floor}{\lfloor}{\rfloor}
\DeclarePairedDelimiter{\ceil}{\lceil}{\rceil}
\DeclareMathOperator{\E}{\mathbf{E}}
\DeclareMathOperator{\Q}{\mathbb{Q}}
\DeclareMathOperator{\R}{\mathbb{R}}
\DeclareMathOperator{\Z}{\mathbb{Z}}
\DeclareMathOperator{\N}{\mathbb{N}}
% \renewcommand{\theenumi}{\alph{enumi}}
\renewcommand{\theenumi}{\roman{enumi}}

\setcounter{secnumdepth}{0}

\rhead{Harry Sha}
\lhead{\today}
\rfoot{Page \thepage}

\title{\textcolor{sectioncolor}{Pupil Tutorial}}
\author{Harry Sha}
\date{\today}

\begin{document}
\clearpage\maketitle
Please contact harry2@stanford.edu with any problems/questions about this pipeline.
\tableofcontents
\thispagestyle{empty}

\sectionfont{\color{sectioncolor}}
\subsectionfont{\color{subsectioncolor}}

\section{Setting Up}
\begin{enumerate}
\item Download python 3.6 from anaconda from this link: \href{https://www.anaconda.com/download/}{https://www.anaconda.com/download/}
\item Follow this link: \href{https://github.com/harrysha1029/pupil}{https://github.com/harrysha1029/pupil} and clone the repository.
\item Put data in a convenient folder, preferably the same one, especially if you plan to run multiple files at the same time.
\item Change behavioral data into the events format
\end{enumerate}

\section{Events Format}
Behavioral data is used to epoch the pupil data. For this to work, the pupil pipeline requires this data to be of a specific format which is described in this section.

The behavioral file should be a MATLAB (.mat) file containing a variable named first, and another variable named events. first should be a float marking the time of the first onset of the behavioral data. This is aligned to



\section{Parameters}
The


\section{Running}
To run the code:
\begin{enumerate}
\item Open terminal (mac, linux) or command prompt (windows)
\item cd into the pipeline
\item \begin{lstlisting}
python pupil_pipeline.py <parameters>
\end{lstlisting}
where $<$parameters$>$ is replaced with the name of your parameters file.
\end{enumerate}If no parameters file is provided, the pipeline will read from parameters\_default.py

\section{Outputs}

\section{Merging}

\section{Explanation of interpolation method}

\end{document}
